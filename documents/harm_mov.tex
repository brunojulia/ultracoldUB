\section{Movimiento oscilatorio arm\'onico}

\subsection{Introducci\'on}
Uno de los movimientos m\'as presentes a nivel f\'isico, incluso en nuestra vida cotidiana, es el movimiento oscilatorio arm\'onico. A nivel fundamental tenemos los muelles o p\'endulos. Qu\'e pasar\'a si analizamos este tipo de movimiento des de el punto de vista de la Mec\'anica Cu\'antica?

\subsection{Simulaci\'on}
Para empezar con la simulaci\'on pulsamos el bot\'on $Demo 2$. Una vez finalizado el periodo de c\'alculo veremos una ventana central muy similar a la del apartado $Movimiento libre$, donde la gr\'afica tiene el mismo significado. 

\begin{itemize}
	\item \textbf{¿Te atreves?} (Nivel bajo): Dado que estamos bajo un movimiento oscilatorio armónico. Sabr\'ias decir donde estar\'a centrada la distribuci\'on 'azul' cuando la 'amarilla' est\'e en un extremo? Y al rev\'es? 
	
	Pista: recuerda de la secci\'on anterior el significado de estas distribuciones.
\end{itemize}

\begin{itemize}
	\item \textbf{¿Te atreves?} (Nivel medio): Ahora que has empezado a notar la presencia del movimiento oscilatorio arm\'onico, vamos a dar un paso m\'as. Entra en la secci\'on $ENERGY$ y observa el gr\'afico. ¿Est\'a en coherencia este gr\'afico con la observaci\'on que has hecho antes? ¿ Cu\'ando ser\'a m\'axima la velocidad y cu\'ando m\'inima?
\end{itemize}

\begin{itemize}
	\item \textbf{¿Te atreves?} (Nivel alto): Ahora que finalmente estamos viendo un claro s\'imil con el movimiento oscilatorio arm\'onico entra en $MEAN VALUE X$. C\'entrate en las lineas: $R-Space <x>$ y $K-Space <k>$. \'Estas representan el valor medio de la posici\'on de nuestra part\'icula y el valor medio de su velocidad, respectivamente. Sabiendo esto, sabr\'ias dar una forma funcional a estas dos variables?
\end{itemize}

Ahora finalmente iremos a observar una analog\'ia cl\'asica con nuestro problema. Como ya se ha comentado al principio uno de los movimientos con los que puede estar uno m\'as familiarizado es el de un muelle, ve\'amos la similitud entre \'este y nuestra part\'icula.
\
Entremos en el apartado $Guess Function$. En \'el, tendremos la capacidad de simular un muelle.

\begin{itemize}
	\item \textbf{¿Te atreves?} (Nivel bajo): Observa que tienes dos par\'ametros a introducir para empezar la simulaci\'on ($A$ y $w$). Sabr\'ias explicar el significado de ambos? Puedes ajustar finalmente los par\'ametros de tal forma de aproximar nuestro muelle al movimiento de nuestra part\'icula?
\end{itemize}

Felicidades! Acabas de comprobar que el movimiento oscilatorio arm\'onico des del punto de la Mec\'anica Cu\'antica, es totalmente an\'alogo al dado por la cl\'asica. Sin embargo, recuerda que aqu\'i estamos hablando de medias, por lo tanto, al hacer un experimento con una part\'icula podemos encontrarla en toda la zona sombreada de 'azul', al igual que con una velocidad en la zona 'amarilla'. Esto significa que a la hora de realizar un experimento con una part\'icula, posiblemente veamos que cuando la detectemos no est\'a en un punto dado por la trayectoria del movimiento arm\'onico, ahora bien, si repetimos el experimento una cantidad de veces muy elevada, en media, tendremos que nuestra part\'icula se sit\'ua en los puntos dados por el movimiento del oscilador arm\'onico.