\section{P\'endulo de Newton Cu\'antico}

\subsection{Introducci\'on}
Como ya hemos comentado anteriormente, en esta secci\'on continuaremos analizando los 'dark soliton' y veremos hasta donde puede llegar su descripci\'on como part\'icula. 
\\

Antes de nada introduciremos lo que es un p\'endulo de Newton. Es un sistema formado por varios p\'endulos; donde unos se sit\'uan en el centro de tal forma que no oscilan y forman una cadena y, otros se desplazan de su posici\'on de equilibrio (en la misma direcci\'on que la cadena). Cuando llega el momento de impacto entre ellos (si tienen la misma masa todos) observamos que sale un n\'umero de p\'endulos, igual que los que han impactado, eyectado por el otro lado de impacto. As\'i sucesivamente se va repitiendo. 
\\

Vamos a recrear este mismo sistema con nuestros 'dark solitons'. Pongamos una serie de ellos en el centro de tal forma que est\'en estacionarios y, pongamos otros descentrados los cu\'ales se empezar\'an a mover hacia los otros estacionarios. Veamos que pasa!

\subsection{Simulaci\'on}
Seleccionamos el bot\'on $Demo 2$ y esperamos a que finalice el c\'alculo. Puedes observar la simulaci\'on utilizando los tres botones ya mencionados anteriormente, pero es posible que cueste apreciar lo que est\'a sucediendo. Si entramos en la secci\'on $DENSITY MAP$, donde vemos la proyecci\'on de nuestro movimiento en el tiempo. Podemos empezar a observar algo curioso en el instante donde se encuentran los 'dark soliton'. Vayamos directamente a $Guess Function$ donde vamos a ver algo asombroso, la similitud con un p\'endulo de Newton.

\begin{itemize}
	\item \textbf{¿Te atreves?} (Nivel bajo): Los par\'ametros a controlar ahora son diversos. $M$ es el n\'umero de bolas que inicialmente oscilan; $N$ es el n\'umero de bolas que inicialmente est\'an parados y $r$ es el radio de las bolas, los dem\'as par\'ametros deber\'ian estar claros. Podr\'ias ajustar los par\'ametros de tal forma de hacer coincidir los dos movimientos (en este caso no ser\'a posible coincidir a la perfecci\'on los dos movimientos, simplemente intenta que sean los m\'as iguales que puedas).
\end{itemize}

\begin{itemize}
	\item \textbf{¿Te atreves?} (Nivel muy alto): Ser\'ias capaz de intuir como debe variar el par\'ametro $r$ en funci\'on de lo lejos que se pongan los solitones que van a impactar con los estacionarios? Augmentar\'a al alejarlos o disminuir\'a? Por qu\'e?
	
	Pista: Si has hecho el m\'odulo de $Bright Solitons$ recuerda el efecto t\'unel.
\end{itemize}

Estamos viendo algo incre\'ible. Los 'dark solitons' se comportan hasta tal punto como part\'iculas que obedecen en gran parte la misma din\'amica que un p\'endulo de Newton, en cierta manera est\'an chocando entre ellos, recordemos que hab\'iamos entendido los 'dark soliton' como 'agujeros'.
\\

As\'i para concluir, simplemente dar una reflexi\'on.Tenemos un sistema formado por un conjunto de part\'iculas en el 0 absoluto de temperatura, confinado por un potencial arm\'onico y al cual se le ha practicado una serie de 'agujeros' (dark solitons). Un sistema realmente complejo el cual hemos llegado a concluir que tiene una clara analog\'ia con el movimiento tan 'simple' como es el P\'endulo de Newton. Realmente asombroso.